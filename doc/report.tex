\documentclass[a4paper]{article}
\usepackage[polish]{babel}
\usepackage[T1]{fontenc}
\usepackage[utf8]{inputenc}
\usepackage{amsfonts}
\usepackage{amsthm}
\usepackage{amsmath}
\usepackage{graphicx}
\usepackage{textcomp}
\usepackage{color}
\usepackage{multirow}
\usepackage{graphics}
\usepackage{listings}

\newtheorem{theorem}{Twierdzenie}

\title{Biologia obliczeniowa}
\author{Mateusz Cicheński (84780 gr.I1) \\ Tomasz Żurkowski (84915 gr.I1)}

\begin{document}
\maketitle
\tableofcontents

\newpage

\section{Wprowadzenie do zagadnienia}

Problem sekwencjonowania łańcuchów DNA jest w ogólności problemem silnie NP-trudnym. W problemie tym mamy dany zbiór {\bf S} słów (oligonukleotydów) o jednakowej długości {\bf l} nad alfabetem {\bf \{A, C, G, T\}}. Ponadto mamy podaną długość sekwencji oryginalnej {\bf n}. Celem sekwencjonowania jest odtworzenie oryginalnego łańcucha DNA, który został poddany hybrydyzacji, na podstawie powyżej zdefiniowanych danych wejściowych.
Dla ułatwienia dalszego opisu wprowadźmy kilka terminów: odległość między oligonukleotydami.
//TODO więcej terminów?
Odległość między oligonukleotydami definiuje się między każdą parą słów jako najmniejsze przesunięcie jednego oligonukletydu względem drugiego, które umożliwi połączenie ich w jeden łańcuch. Dla przykładu słowo $ACGTA$ znajduje się w odległości 2 względem słowa $ACACG$ (relacja w drugą stronę - słowo $ACACG$ znajduje się w odległości 4 względem słowa $ACGTA$).
W zaproponowanych przez nas podejściach będziemy wykorzystywać teorię grafów w celu szybkiego łączenia oligonukleotydów oddalonych od siebie o jak najmniejszą wartość.

W analizie problemu można wyróżnić cztery szczególne przypadki - idealny, z błędami negatywnymi, z błędami pozytywnymi, z obydwoma typami błędów.
Przypadek idealny to taki, gdzie na mikromacierzy dopasowały się wszystkie słowa występujące w sekwencji.
Mówimy, że dane odwzorowanie łańcucha DNA na mikromacierzy zawiera błędy negatywne, jeśli z jakiegoś powodu niektóre słowa nie zostały dopasowane do macierzy. Otrzymujemy więc niepełne spektrum sekwencji.
Błędami pozytywnymi określamy nadmiar informacji, tzn. na mikromacierzy zostały dopasowane słowa, które nie występują w sekwencji oryginalnej. Mamy więc zbiór słów które są nadzbiorem zbioru słów wchodzących w skład oryginalnej sekwencji.

\section{SBH with positive errors}
Jak już wcześniej mówiliśmy, sekwencjonowanie łańcucha DNA z błędami pozytywnymi charakteryzuje się nadmiarową ilością słów otrzymanych w spektrum. Mamy więc pewność, że pośród tych słów znajdzie się dokładnie $n-l+1$ z których będziemy mogli zrekonstruować oryginalną sekwencję.
W tym celu proponujemy następujący algorytm:
1. Utwórz graf skierowany {\bf G}. Niech zbiór wierzchołków V stanowią słowa tworzące spektrum sekwencji DNA. Dla każdej pary wierzchołków $(i,j)$ dodaj łuk od wierzchołka $i$ do $j$ wtw (wtw = wtedy i tylko wtedy, nie wiem czy to skrót powszechny) gdy słowo w wierzchołku $j$ jest oddalone o 1 od słowa w wierzchołku $i$
2. Znajdź silnie spójną składową grafu {\bf SCC}
3. Uszereguj topologicznie {\bf SCC}
4. Dla każdego wierzchołka znajdź ścieżkę przechodzącą przez n-l+1 wierzchołków
5. Wybierz ścieżkę optymalną ścieżkę (//TODO jak to się dzieje?) 

\section{Wyniki}

\section{Bu}


\end{document}
