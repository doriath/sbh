\documentclass[a4paper,10pt]{article}
\usepackage[pdftex]{graphicx}
\usepackage{polski}
\usepackage[utf8]{inputenc}
\usepackage{textcomp}
\usepackage{pifont}
\usepackage{color}
\usepackage{listings}

% opcje listingów
\definecolor{darkgray}{rgb}{0.95,0.95,0.95}
\lstset{language=c++}
\lstset{backgroundcolor=\color{darkgray}}

\title{\textbf{Biologia obliczeniowa}}
\author{\textbf{Mateusz Cicheński} [84780, I1]\\ \textbf{Tomasz Żurkowski} [84915, I1]}
\date{24 kwietnia 2010 r.}

\begin{document}
 
\maketitle

\tableofcontents

\newpage

\section{Wprowadzenie do zagadnienia}

Problem sekwencjonowania łańcuchów DNA jest w ogólności problemem silnie NP-trudnym. W problemie tym mamy dany zbiór {\bf S} słów (oligonukleotydów) o jednakowej długości {\bf l} nad alfabetem {\bf \{A, C, G, T\}}. Ponadto mamy podaną długość sekwencji oryginalnej {\bf n}. Celem sekwencjonowania jest odtworzenie oryginalnego łańcucha DNA, który został poddany hybrydyzacji, na podstawie powyżej zdefiniowanych danych wejściowych.
Dla ułatwienia dalszego opisu wprowadźmy kilka terminów: odległość między oligonukleotydami.
%TODO więcej terminów?
Odległość między oligonukleotydami definiuje się między każdą parą słów jako najmniejsze przesunięcie jednego oligonukletydu względem drugiego, które umożliwi połączenie ich w jeden łańcuch. Dla przykładu słowo $ACGTA$ znajduje się w odległości 2 względem słowa $ACACG$ (relacja w drugą stronę - słowo $ACACG$ znajduje się w odległości 4 względem słowa $ACGTA$).
W zaproponowanych przez nas podejściach będziemy wykorzystywać teorię grafów w celu szybkiego łączenia oligonukleotydów oddalonych od siebie o jak najmniejszą wartość.

W analizie problemu można wyróżnić cztery szczególne przypadki - idealny, z błędami negatywnymi, z błędami pozytywnymi, z obydwoma typami błędów.
Przypadek idealny to taki, gdzie na mikromacierzy dopasowały się wszystkie słowa występujące w sekwencji.
Mówimy, że dane odwzorowanie łańcucha DNA na mikromacierzy zawiera błędy negatywne, jeśli z jakiegoś powodu niektóre słowa nie zostały dopasowane do macierzy. Otrzymujemy więc niepełne spektrum sekwencji.
Błędami pozytywnymi określamy nadmiar informacji, tzn. na mikromacierzy zostały dopasowane słowa, które nie występują w sekwencji oryginalnej. Mamy więc zbiór słów które są nadzbiorem zbioru słów wchodzących w skład oryginalnej sekwencji.

\section{SBH z błędami pozytywnymi}
Jak już wcześniej mówiliśmy, sekwencjonowanie łańcucha DNA z błędami pozytywnymi charakteryzuje się nadmiarową ilością słów otrzymanych w spektrum. Mamy więc pewność, że pośród tych słów znajdzie się dokładnie $n-l+1$ z których będziemy mogli zrekonstruować oryginalną sekwencję.

Wprowadźmy na początku pomocniczy algorytm usuwania cykli z grafu $G$:

Algorytm 1:
\begin{enumerate}
 \item Znajdź silnie spójne składowe grafu $G$
 \item Jeśli w grafie nie ma silnie spójnych składowych, zakończ algorytm
 \item Uszereguj je topologicznie, traktując każdą silnie spójną składową jako wierzchołek oraz tworząc łuk z silnie spójnej składowej $SCC1$ do $SCC2$ jeśli jakikolwiek wierzchołek z $SCC1$ łączy się z dowolnym wierzchołkiem z $SCC2$
 \item Dla danej silnie spójnej składowej $SCC$ znajdź należący do niej wierzchołek $k$, który łączy się z pozostałymi silnie spójnymi składowymi lub weź dowolny wierzchołek jeśli nie ma połączeń z innymi silnie spójnymi grafu
 \item Usuń wszystkie połączenia wychodzące z $k$, które łączą się z innymi wierzchołkami wchodzącymi w skład $SCC$
 \item Idź do punktu 1
\end{enumerate}

Mając do dyspozycji algorytm usuwania cykli w grafie proponujemy następujący algorytm rozwiązujący problem sekwencjonowania DNA z błędami pozytywnymi:

Algorytm 2:
\begin{enumerate}
 \item Utwórz graf skierowany {\bf G}. Niech zbiór wierzchołków V stanowią słowa tworzące spektrum sekwencji DNA. Dla każdej pary wierzchołków $(i,j)$ dodaj łuk od wierzchołka $i$ do $j$ wtw gdy słowo w wierzchołku $j$ jest oddalone o 1 od słowa w wierzchołku $i$
 \item Usuń cykle w grafie {\bf G} tworząc graf {\bf RES} korzystając z Algorytmu 1.
 \item Uszereguj topologicznie {\bf RES}
 \item Wybierz najdłuższą możliwą ścieżkę w grafie {\bf RES}
\end{enumerate}

Złożoność powyższego algorytmu to $ $. Złożoność poszczególnych punktów jest następująca:
\begin{itemize}
 \item $S^2$ dla punktu 1,
 \item $S + A$ dla punktu 2.1, gdzie liczba łuków jest zależna od słów tworzących spektrum i na pewno nie większa niż $S * (S-1) / 2$
 \item $S + A$ dla punktu 2.2
\end{itemize}

\section{SBH general}
W przypadku podejścia ogólnego nie mamy gwarancji, że istnieje ścieżka po łukach o koszcie 1, która utworzy sekwencję odpowiedniej długości. Musimy więc uwzględnić mniej korzystne połączenia o kosztach 2 i wyższych. W związku z tym wykorzystamy algorytm sekwencjonowania z błędami pozytywnymi uogólniając jego działanie.

Algorytm po modyfikacji wygląda następująco
\begin{enumerate}
 \item Wczytaj zbiór wszystkich słów tworzących spektrum do słownika OLIGS
 \item Utwórz graf skierowany {\bf G}. Niech zbiór wierzchołków V stanowią słowa w słowniku OLIGS. Dla każdej pary wierzchołków $(i,j)$ dodaj łuk od wierzchołka $i$ do $j$ o koszcie równym odległości słowa w wierzchołku $j$ względem słowa w wierzchołku $i$
 \item Znajdź najniższy koszt łuku MIN w grafie {\bf G}
 \item Utwórz graf skierowany {\bf G'}, który jest podgrafem grafu {\bf G} i zawiera tylko łuki o koszcie MIN
 \item Usuń cykle w grafie {\bf G'} tworząc graf {\bf RES} korzystając z Algorytmu 1
 \item Uszereguj topologicznie {\bf RES}
 \item Wybierz najdłuższą możliwą ścieżkę {\bf PATH} w grafie {\bf RES}
 \item Usuń ze słownika OLIGS słowa wchodzące w skład ścieżki {\bf PATH}, w zamian dodając słowo utworzone przez połączenie słów zgodnie z kolejnością w ścieżce {\bf PATH}
 \item Jeśli słownik OLIGS zawiera dokładnie jedno słowo, to zakończ, w przeciwnym wypadku idź do punktu 2
\end{enumerate}

Ideą algorytmu jest wyszukanie teoretycznie najkorzystniejszych połączeń o koszcie 1 tworzących sekwencje pełne (tj. w ich skład wchodzą tylko istniejące słowa), a jeśli nie uda się utworzyć na ich podstawie sekwencji wyjściowej, to połączenie sekwencji utworzonych w ten sposób uwzględniając brakujące słowa, czyli łącząc łukami o koszcie 2 i wyższymi. 

Przykład:
Dane wejściowe:
$OLIGS = { AGC, GCT, CTG, TGT, GTG, TGA, GAC, ACT }$
$n = 10$

%               ----------------------
%               |                    v
%AGC -> GCT -> CTG -> TGT -> GTG -> TGA -> GAC -> ACT
%               ^      ^      |                    |
%               |      --------                    |
%               ------------------------------------

Powyżej zaprezentowano graf utworzony w kroku 4 dla wartości $MIN=1$. Wcześniejsze kroki są jednoznacznie interpretowalne, a utworzenie pełnego grafu z punktu 2 bardzo zamgliłoby obraz grafu. Usuńmy teraz cykle.
Jak widać w powyższym grafie mamy następujące silnie spójne składowe:
$A$ - {AGC}
$B$ - {GCT}
$C$ - {CTG, TGT, GTG, TGA, GAC, ACT}
Musimy więc rozbić wszystkie silnie spójne składowe usuwając cykle, co umożliwi nam utworzenie nadłuższej możliwej ścieżki od ustalonego punktu.
Uzyskaliśmy tylko jedną silnie spójną składową $C$, która posiada więcej niż jeden wierzchołek, więc ją należy rozbić. Zauważmy, że po uszeregowaniu topologicznie spójnych składowych uzyskamy następujące uporządkowanie: $A->B->C$, a więc bierzemy pierwszy lepszy wierzchołek silnie spójnej składowej C i usuwamy z niego wszystkie łuki. Ponieważ wybieramy dowolny wierzchołek, załóżmy, że jest to wierzchołek $TGT$, usuwamy zatem łuk prowadzący do $GTG$.
Ponownie generujemy silnie spójne składowe grafu:
$A$ - {AGC}
$B$ - {GCT}
$C$ - {CTG, TGA, GAC, ACT}
$D$ - {TGT}
$E$ - {GTG}
Ponieważ znów mamy silnie spójną składową, która ma więcej niż jeden wierzchołek, to z niej będziemy usuwać cykl. Tym razem jednak po uszeregowaniu topologicznie silnie spójnych składowych otrzymujemy zależność $A->B->C->D->E$. Oznacza to, że w silnie spójnej składowej C musimy odnaleźć wierzchołek prowadzący do silnie spójnej składowej $D$. Takim wierzchołkiem jest $CTG$, usuwamy więc z niego wszystkie łuki wychodzące z niego, które prowadzą do innych wierzchołków silnie spójnej składowej $C$. Zostanie usunięty łuk $CTG->TGA$.
W kolejnej iteracji podziału na silnie spójne składowe uzyskamy siedem jedno-elementowych zbiorów, a więc usuwanie cykli możemy zakończyć. Po wykonaniu powyższych operacji graf wygląda następująco:

%AGC -> GCT -> CTG -> TGT    GTG -> TGA -> GAC -> ACT
%               ^      ^      |                    |
%               |      --------                    |
%               ------------------------------------

Po uszeregowaniu topologicznie możemy uzyskać m.in. następującą kolejność wierzchołków:
$AGC, GCT, GTG, TGA, GAC, ACT, CTG, TGT$
Szukamy najdłuższej możliwej ścieżki korzystając z uporządkowania topologicznego:
$GTG, TGA, GAC, ACT, CTG, TGT$
I tworzymy nowe słowo sklejając poszczególne elementy ze sobą w sekwencję $GTGACTGT$.
Usuwamy wierzchołki, z których utworzyliśmy powyższą sekwencją. Dodajemy otrzymaną sekwencję do grafu, który obecnie wygląda następująco:

AGC -> GCT         GTGACTGT

Uwzględnione są tylko łuki o koszcie $MIN=1$ (czyli niejako uzyskaliśmy las grafów). Ponieważ nie ma tu silnie spójnych składowych możemy przystąpić do łączenia elementów. Najdłuższą ścieżką jest $AGC, GCT$ (liczy się ilość wierzchołków wchodzących w skład ścieżki, nie łączna długość słów!), zatem tą sekwencję łączymy i dodajemy do grafu usuwając wykorzystane do jej utworzenia wierzchołki.
Otrzymujemy wówczas poniższy graf:

%         4
%  ----------------
%  |              v
%AGCT          GTGACTGT
%  \^              |
%  ----------------
%         8
		 
Jest to graf pełny, nad łukami zaznaczono ich koszty. Ponieważ najniższym kosztem jest $MIN=4$, oznacza to, że nie istnieje już żadne bezpośrednie połączenie, musimy połączyć te sekwencje "grubymi nićmi" aby utworzyć sekwencję końcową:
$AGCTGTGACTGT$
Zauważmy, że jest ona dłuższa, niż $n=10$. W związku z tym sekwencję wynikową przycinamy do długości odpowiadającej spodziewanemu wynikowi otrzymując w rezultacie:
$AGCTGTGACT$
Algorytm zakończył swoje działanie. Zupełnie przypadkowo uzyskana sekwencja zgadza się z sekwencją wyjściową, mimo tego, że w ostatnim kroku algorytmu zostały sklejone słowa w sposób niekorzystny.
Należy zauważyć, że znaczny wpływ na działanie algorytmu ma sposób wyboru wierzchołka w silnie spójnej składowej grafu z którego zostaną usunięte łuki, a także kolejność występowania wierzchołków w silnie spójnych składowych grafu. Jeśli na początku zamiast usuwać łuk z wierzchołka $TGT$ usunęlibyśmy dowolny inny łuk z tej silnie spójnej składowej grafu, to dalsze iteracje miałyby inny przebieg.


\section{Wyniki}

\section{Bu}


\end{document}
